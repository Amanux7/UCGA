\documentclass[11pt]{article}

\usepackage{amsmath}
\usepackage{amssymb}
\usepackage{graphicx}
\usepackage{geometry}
\usepackage{float}
\usepackage{listings}
\usepackage{xcolor}
\usepackage{hyperref}

\geometry{margin=1in}

% ======================
% CODE STYLE
% ======================

\definecolor{codegray}{gray}{0.95}

\lstset{
backgroundcolor=\color{codegray},
basicstyle=\ttfamily\small,
breaklines=true,
language=Python
}

% ======================
% TITLE
% ======================

\title{
Unified Cognitive Graph Architecture (UCGA): \\
A Graph-Native Cognitive Framework for Artificial General Intelligence, AI Agents, and Multimodal Systems
}

\author{
Aman Singh \\
Founder, UCGA Research Initiative
}

\date{2026}

\begin{document}

\maketitle

% ======================
% ABSTRACT
% ======================

\begin{abstract}

This paper introduces Unified Cognitive Graph Architecture (UCGA), a novel graph-native cognitive framework designed to address fundamental limitations of traditional neural architectures in reasoning, memory integration, and adaptive intelligence. UCGA models intelligence as recursive interaction between cognitive nodes connected through adaptive weighted edges. We present the architecture, mathematical formulation, prototype implementation, and initial experimental validation.

We further outline a future roadmap for scaling UCGA into a full artificial general intelligence system capable of multimodal reasoning, adaptive learning, and autonomous agent operation.

\end{abstract}

% ======================
% INTRODUCTION
% ======================

\section{Introduction}

Artificial intelligence systems today rely heavily on sequential neural architectures such as feedforward networks and transformer models. While these systems achieve strong performance in specific domains, they remain fundamentally limited in recursive reasoning, persistent memory integration, and adaptive cognitive restructuring.

Biological intelligence operates differently. Cognitive processes emerge from dynamic interaction between specialized regions forming a graph-like structure.

This paper introduces Unified Cognitive Graph Architecture (UCGA), which models intelligence as a dynamic cognitive graph.




% ======================
% PROBLEM STATEMENT
% ======================

\section{Problem Statement}

Current architectures face several key limitations:

\begin{itemize}

\item Static architecture structure

\item Limited recursive reasoning capability

\item Weak persistent memory integration

\item Poor cross-domain adaptability

\end{itemize}

UCGA addresses these limitations by modeling intelligence as a recursive cognitive graph.

% ======================
% ARCHITECTURE
% ======================

\section{UCGA Architecture}

UCGA defines intelligence as:

\begin{equation}
G = (V, E, W, S)
\end{equation}

Where:

\begin{itemize}

\item V = cognitive nodes

\item E = connections

\item W = adaptive weights

\item S = cognitive state

\end{itemize}

\begin{figure}[H]
\centering
\includegraphics[width=\linewidth]{figure1_ucga_architecture.png}
\caption{Unified Cognitive Graph Architecture}
\end{figure}

% ======================
% RECURSIVE LOOP
% ======================

\section{Recursive Cognitive Refinement}

UCGA operates through recursive refinement cycles.

\begin{figure}[H]
\centering
\includegraphics[width=\linewidth]{figure2_cognitive_loop.png}
\caption{Recursive Cognitive Refinement Loop}
\end{figure}

% ======================
% MATHEMATICS
% ======================

\section{Mathematical Formulation}

State update equation:

\begin{equation}
v_i(t+1) =
\sigma
\left(
\sum_j W_{ij} v_j(t)
+
b_i
\right)
\end{equation}

Connection update equation:

\begin{equation}
W_{ij}(t+1) =
W_{ij}(t)
+
\alpha \delta_{ij}(t)
\end{equation}

Convergence condition:

\begin{equation}
\| S(t+1) - S(t) \| < \epsilon
\end{equation}

\begin{figure}[H]
\centering
\includegraphics[width=\linewidth]{figure3_math_model.png}
\caption{Mathematical interaction model}
\end{figure}

% ======================
% IMPLEMENTATION
% ======================

\section{Prototype Implementation}

UCGA has been implemented using PyTorch.

\begin{lstlisting}

class CognitiveNode(nn.Module):

    def __init__(self, dim):
        super().__init__()
        self.linear = nn.Linear(dim, dim)

    def forward(self, inputs):
        combined = sum(inputs)
        return torch.tanh(self.linear(combined))

\end{lstlisting}

% ======================
% EXPERIMENTS
% ======================

\section{Experimental Validation}

Initial experiments were conducted on synthetic reasoning tasks.

Results demonstrate:

\begin{itemize}

\item Stable recursive convergence

\item Adaptive reasoning improvement

\item Memory-assisted reasoning

\end{itemize}

These results validate UCGA core principles.

% ======================
% COMPARISON
% ======================

\section{Comparison with Traditional Architectures}

\begin{figure}[H]
\centering
\includegraphics[width=\linewidth]{figure4_comparison.png}
\caption{UCGA vs Traditional Architectures}
\end{figure}

UCGA differs fundamentally from layer-based architectures by modeling intelligence as cognitive graph.

% ======================
% CURRENT STATUS
% ======================

\section{Current Development Status}

UCGA currently includes:

\begin{itemize}

\item Formal architecture definition

\item Mathematical formulation

\item Prototype implementation

\item Initial experimental validation

\item Language reasoning prototype

\item Persistent memory integration

\end{itemize}

% ======================
% FUTURE WORK
% ======================

\section{Future Work and Roadmap}

Future research will focus on:

\begin{itemize}

\item Large-scale training on language datasets

\item Multimodal integration (vision, language, audio)

\item Autonomous reasoning agents

\item Self-improving cognitive graph evolution

\item Real-world deployment and scaling

\end{itemize}

These developments aim to scale UCGA toward full artificial general intelligence capability.

% ======================
% LIMITATIONS
% ======================

\section{Current Limitations}

Current prototype has limitations:

\begin{itemize}

\item Limited training scale

\item Small experimental datasets

\item Prototype-level implementation

\end{itemize}

These limitations will be addressed in future work.

% ======================
% AI ASSISTANCE DISCLOSURE
% ======================

\section{AI Assistance and Development Tools}

During the development of the Unified Cognitive Graph Architecture (UCGA), modern AI-assisted research and development tools were utilized to support ideation, implementation structuring, documentation refinement, and prototype development.

Specifically, large language models including Gemini (Google) and GPT-5.2 (OpenAI) were used as cognitive assistance tools to help accelerate:

\begin{itemize}

\item Research structuring and conceptual clarification

\item Mathematical formulation refinement

\item Prototype implementation guidance

\item Documentation and technical writing support

\item Development workflow planning

\end{itemize}

These tools functioned strictly as research assistants and development aids. All architectural design decisions, conceptual framework definition, and research direction were conceived, evaluated, and directed by the author.

The Unified Cognitive Graph Architecture itself is an original architectural framework proposed and developed by the author.



% ======================
% CONCLUSION
% ======================

\section{Conclusion}

UCGA introduces a new paradigm for artificial intelligence architecture based on cognitive graph interaction. Initial results demonstrate feasibility and potential for scaling toward general intelligence systems.

Future development will focus on large-scale training, multimodal integration, and autonomous reasoning.

% ======================
% ACKNOWLEDGMENTS
% ======================

\section{Acknowledgments}

The author acknowledges the use of modern AI-assisted development tools including Gemini (Google) and GPT-5.2 (OpenAI) for providing technical assistance during the research and development process.

These tools were used to support implementation, documentation, and development acceleration.

The author also acknowledges the broader open-source research community whose foundational work in neural networks, graph neural networks, and cognitive architectures has contributed to the scientific ecosystem enabling this research.

% ======================
% END
% ======================

\end{document}
