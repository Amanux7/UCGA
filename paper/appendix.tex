% ============================================================================
%  Appendix — Key Implementation Details & Hyperparameters
% ============================================================================

\section{Full Node Interface Specification}
\label{sec:node_interfaces}

\Cref{tab:node_interfaces} provides a complete specification of input/output interfaces for all cognitive nodes in \UCGA{}.

\begin{table*}[t]
  \centering
  \caption{Complete node interface specification for all \UCGA{} cognitive nodes.}
  \label{tab:node_interfaces}
  \small
  \begin{tabular}{@{}lllll@{}}
    \toprule
    \textbf{Node} & \textbf{Input(s)} & \textbf{Output} & \textbf{Parameters} & \textbf{Special Mechanism} \\
    \midrule
    PerceptionNode  & Raw encoded input $\bx$                         & $\bv_{\text{perc}} \in \Real^d$  & $\bW_p, \mathbf{b}_p$            & GELU + LayerNorm \\
    MemoryNode      & $\bv_{\text{perc}}$, memory bank $\bM$          & $\bv_{\text{mem}} \in \Real^d$   & $\bW_Q$, $\bW_{\text{agg}}$      & Scaled dot-product attention \\
    ReasoningNode   & $\bv_{\text{perc}}$, $\bv_{\text{mem}}$          & $\bv_{\text{reas}} \in \Real^d$  & $\bW_r^{(1..K)}$, $\mathbf{b}_r$ & $K$-step residual refinement \\
    PlanningNode    & $\bv_{\text{reas}}$                              & $\bv_{\text{plan}} \in \Real^d$  & $\bW_{\text{plan}}$               & Subgoal embedding \\
    EvaluationNode  & $\bv_{\text{plan}}$, $\bv_{\text{reas}}$         & $c \in [0,1]$, $\bv_{\text{eval}} \in \Real^d$ & $\bW_c$, $b_c$ & Sigmoid confidence gate \\
    CorrectionNode  & $\bv_{\text{plan}}$, $\bv_{\text{eval}}$         & $\bv_{\text{corr}} \in \Real^d$  & $\bW_g$, $\bW_{\text{corr}}$      & Gated residual correction \\
    BalancerNode    & $\bv_{\text{reas}}$, $\bv_{\text{corr}}$, $\bv_{\text{mem}}$ & $\bv_{\text{bal}} \in \Real^d$ & $\bW_\alpha$ & Softmax-weighted fusion \\
    OutputNode      & $\bv_{\text{bal}}$                               & $\hat{y} \in \Real^C$            & $\bW_{\text{out}}^{(1,2)}$        & Residual bypass + MLP \\
    PersistentMemory & $\bv_{\text{bal}}$ (write), $\bq$ (read)        & Retrieved $\bv \in \Real^d$      & $\bW_Q$                           & Least-used-slot write \\
    \bottomrule
  \end{tabular}
\end{table*}

% ============================================================================

\section{Python Implementation Excerpts}
\label{sec:code_appendix}

\subsection{Base Cognitive Node}

The following listing shows the core \texttt{CognitiveNode} class with GELU + LayerNorm activation, which all specialised nodes inherit from:

\begin{lstlisting}[caption={CognitiveNode base class (GELU + LayerNorm).}]
class CognitiveNode(nn.Module):
    def __init__(self, input_dim, state_dim,
                 name="CognitiveNode"):
        super().__init__()
        self.name = name
        self.input_dim = input_dim
        self.state_dim = state_dim
        self.W = nn.Linear(input_dim, state_dim,
                           bias=False)
        self.b = nn.Parameter(torch.zeros(state_dim))
        self.activation = nn.GELU()
        self.norm = nn.LayerNorm(state_dim)
        self.register_buffer(
            "state", torch.zeros(1, state_dim))

    def forward(self, inputs):
        aggregated = torch.stack(
            inputs, dim=0).sum(dim=0)
        self.state = self.norm(
            self.activation(
                self.W(aggregated) + self.b))
        return self.state

    def reset_state(self, batch_size=1):
        with torch.no_grad():
            self.state = torch.zeros(
                batch_size, self.state_dim,
                device=self.b.device)
\end{lstlisting}

\subsection{UCGA Model Cognitive Loop}

\begin{lstlisting}[caption={UCGAModel forward pass (simplified).}]
def forward(self, x, return_meta=False):
    B = x.size(0)
    self._reset_all(B)
    memory_bank = self.persistent_memory \
        .get_memory_bank(B)

    for t in range(self.cognitive_steps):
        percept = self.perception([x])
        mem_state = self.memory_node(
            [percept], memory_bank)
        reason_state = self.reasoning(
            [percept, mem_state])
        plan_state = self.planning(
            [reason_state])
        eval_state = self.evaluation(
            [plan_state, reason_state])
        confidence = self.evaluation \
            .get_confidence()

        if confidence.mean().item() \
                < self.correction_threshold:
            corrected = self.correction(
                [plan_state, eval_state])
        else:
            corrected = plan_state

        balanced = self.balancer(
            [reason_state, corrected, mem_state])
        x = balanced  # recurrent refinement

    output = self.output_node([balanced])
    self.persistent_memory.write(
        balanced.detach())
    return output
\end{lstlisting}

\subsection{Persistent Memory Read}

\begin{lstlisting}[caption={Attention-based memory retrieval.}]
def read(self, query):
    B = query.size(0)
    mem = self._expand_memory(B)
    Q = self.read_query(query).unsqueeze(1)
    scores = torch.bmm(
        Q, mem.transpose(1, 2)) / self.attn_scale
    weights = torch.softmax(scores, dim=-1)
    retrieved = torch.bmm(
        weights, mem).squeeze(1)
    return retrieved
\end{lstlisting}

\subsection{Output Node with Residual Bypass}

\begin{lstlisting}[caption={OutputNode with residual connection.}]
class OutputNode(CognitiveNode):
    def __init__(self, state_dim, output_dim):
        super().__init__(
            input_dim=state_dim,
            state_dim=state_dim)
        self.output_head = nn.Sequential(
            nn.Linear(state_dim, state_dim),
            nn.GELU(),
            nn.Linear(state_dim, output_dim),
        )

    def forward(self, inputs):
        aggregated = torch.stack(
            inputs, dim=0).sum(dim=0)
        h = super().forward(inputs)
        combined = h + aggregated  # residual
        return self.output_head(combined)
\end{lstlisting}

\subsection{Cognitive Agent}

\begin{lstlisting}[caption={CognitiveAgent perceive-and-act cycle.}]
@torch.no_grad()
def perceive_and_act(self, raw_input):
    self.model.eval()
    self.encoder.eval()
    encoded = self.encoder(
        raw_input.to(self.device))
    output, meta = self.model(
        encoded, return_meta=True)
    self.history.append({
        "output_norm": output.norm().item(),
        "confidence": meta["confidences"][-1],
        "corrections": meta["corrections"],
    })
    return {"output": output, "meta": meta}
\end{lstlisting}

% ============================================================================

\section{Encoder Architectures}
\label{sec:encoder_details}

\begin{table}[H]
  \centering
  \caption{Input encoder configurations.}
  \label{tab:encoders}
  \small
  \begin{tabular}{@{}llp{4.5cm}@{}}
    \toprule
    \textbf{Encoder} & \textbf{Input Type} & \textbf{Architecture} \\
    \midrule
    VectorEncoder  & Dense vector  & 2-layer MLP with GELU activation \\
    TextEncoder    & Token indices & Multi-scale 1D CNN (kernel sizes 3, 5, 7) with max pooling \\
    ImageEncoder   & $3 \times 32 \times 32$ & 3-layer CNN with batch normalisation, adaptive average pooling \\
    \bottomrule
  \end{tabular}
\end{table}
